%-----------------------------------------------------------------------------%
\chapter{PENDAHULUAN}
%-----------------------------------------------------------------------------%

\vspace{6pt}

\section{Latar Belakang} \label{sec:latar_belakang}
Dekade ini, kita telah memasuki era informasi, yaitu era di mana informasi mengalir dengan arus yang sangat kencang dan hampir tak terbendung. Kemajuan teknologi telah memungkinkan kita untuk memanfaatkan arus informasi ini. Aplikasi multimedia terkini ditambah dengan kemampuan jaringan internet sekarang perlahan mengarahkan kita untuk bertukar informasi dengan gambar. Dengan adanya tren baru ini, diperlukan adanya mekanisme pengamanan tertentu agar informasi tidak jatuh ke tangan yang salah.

Untuk mengatasi efek negatif tersebut, ilmuwan komputer mengembangkan ilmu kriptografi, yaitu ilmu yang mempelajari prinsip dan metode untuk mengubah tulisan jelas (\textit{clear text}) menjadi tulisan yang tidak bermakna (\textit{cipher text}) dan kemudian dikembalikan lagi ke bentuk aslinya. Seiring dengan perkembangan zaman, kriptografi tidak melulu menyangkut teks, tapi dapat pula diaplikasikan ke jenis informasi lainnya, seperti citra, video, dan suara. Akan tetapi, perkembangan ini tidaklah tanpa biaya. Semakin rumit model informasi yang disandikan, semakin rumit pula algoritma penyandiannya.

Panighray dkk (2008) dalam penelitiannya mengembangkan metode enkripsi citra dengan metode Hill's Cipher yang termodifikasi. Metode ini menggunakan \emph{involutory key matrix}, yaitu matriks yang merupakan invers dirinya sendiri. Algoritma involutory key matrix ini diusulkan oleh Acharya dkk (2008) dalam jurnalnya yang berjudul “\textit{Novel Methods of Generating Self-Invertible Matrix for Hill Cipher Algorithm}”. Metode ini sudah cukup baik untuk menyandikan citra, akan tetapi matriks kuncinya dibuat secara acak. Penulis berusaha mengembangkan metode ini sehingga pengguna dapat menentukan sendiri matriks kuncinya.

Bertolak dari permasalahan di atas, penulis mengusulkan untuk membuat metode enkripsi yang berdasar dari penelitian Panighray dkk. dengan sedikit perubahan, yaitu menggunakan matriks kunci yang ditentukan oleh pengguna sendiri.
%\lipsum[4-5]

\section{Rumusan Masalah}
Berdasarkan latar belakang di atas dapat diidentifikasi masalah-masalah sebagai berikut:
\begin{enumerate}[nolistsep,leftmargin=1cm]
\item Bagaimanakah algoritma penyandian baru yang didasarkan metode Advanced Hill Cipher?
\item Bagaimana kehandalan metode yang diusulkan penulis dibandingkan dengan metode Hill Cipher yang asli?
\item Apakah metode yang diusulkan penulis layak untuk diterapkan untuk penyandian citra? Bagaimana pengujiannya?
\end{enumerate}

\section{Tujuan Penelitian}
\begin{enumerate}[nolistsep,leftmargin=1cm]
\item Menemukan metode penyandian baru berdasar metode Advanced Hill Cipher usulan Panighray dkk.
\item Menguji kehandalan metode yang diusulkan penulis.
\end{enumerate}

\section{Batasan Masalah}
Dalam penelitian ini, peneliti akan membatasi masalah yang akan diteliti antara lain:
\begin{enumerate}[nolistsep,leftmargin=1cm]
\item Citra yang dienkripsi berupa citra berwarna tiga kanal (\textit{RGB}) atau citra keabuan (\textit{grayscale}).
\item Format citra yang diolah adalah PNG, JPG, atau BMP.
\end{enumerate}

\section{Manfaat Penelitian}


\section{Metodologi Penelitian}
\subsection{Studi Literatur}
Studi literatur dilakukan untuk mendapatkan informasi terkait penelitian yang
dilakukan. Studi literatur dilakukan dengan cara mempelajari buku referensi, artikel
dan jurnal yang berkaitan dengan kriptografi dan pemrosesan citra.

\subsection{Percancangan Sistem}
Pada tahap ini dilakukan perancangan pembuatan sistem berdasarkan teori-teori yang didapatkan 
dari tahap sebelumnya.

\subsection{Pengumpulan Data}
Pada tahap ini dilakukan pengumpulan data yang digunakan untuk penelitian.

\subsection{Implementasi Sistem}
Tahap ini merupakan tahap di mana sistem diwujudkan. Sistem dibuat sesuai dengan rancangan 
pada tahap sebelumnya.

\subsection{Pengujian}
Setelah sistem selesai dibuat, data yang sudah terkumpul diuji menggunakan sistem tersebut
untuk kemudian diuji kesahihannya dibandingkan dengan teori yang ada.

\section{Sistematika Penulisan}
\subsection*{BAB I: PENDAHULUAN} %\textbf{BAB I: PENDAHULUAN}
\begin{addmargin}[0.75cm]{0em}
Bab ini berisi tentang latar belakang masalah, rumusan masalah, tujuan penelitian, batasan masalah, manfaat penelitian, metodologi penelitian, dan sistematika penulisan.
\end{addmargin}
\subsection*{BAB II: LANDASAN TEORI}
\begin{addmargin}[0.75cm]{0em}
Bab ini berisi tentang teori-teori yang mendukung atau berhubungan denga aplikasi ini.
\end{addmargin}
\subsection*{BAB III: ANALISA DAN PERANCANGAN SISTEM}
\begin{addmargin}[0.75cm]{0em}
Bab ini menjelaskan tentang proses menganalisa dan merancang sistem aplikasi ini.
\end{addmargin}
\subsection*{BAB IV: IMPLEMENTASI DAN PENGUJIAN}
\begin{addmargin}[0.75cm]{0em}
Bab ini berisi tentang implementasi dan pengujian sistem aplikasi yang telah dibuat.
\end{addmargin}
\subsection*{BAB V: KESIMPULAN DAN SARAN}
\begin{addmargin}[0.75cm]{0em}
Bab ini berisi tentang kesimpulan dan saran untuk mendukung perbaikan sistem aplikasi ini.
\end{addmargin}

\newpage