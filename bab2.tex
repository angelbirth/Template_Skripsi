%-----------------------------------------------------------------------------%
\chapter{LANDASAN TEORI}
%-----------------------------------------------------------------------------%

%
\vspace{12pt}


\section{Kriptografi}
Kriptografi adalah ilmu yang melindungi data dan komunikasi. Objektif utama
kriptografi adalah untuk memungkinkan dua pihak berkomunikasi lewat saluran
tidak aman sedemikian rupa sehingga pihak ketiga tidak dapat memahami apa yang
sedang dibicarakan.\par
Salah satu komponen utamanya melibatkan pertukaran pesan atau informasi antara
dua pihak dengan mengubah wujud pesan/data dengan berbagai cara yang bertujuan
untuk membuat pihak lain kesulitan "menguping" pembicaraan. Aspek lainnya
termasuk otentikasi -- proses meyakinkan penerima pesan bahwa pesan benar-benar
berasal dari sumber yang asli (bukan dari pengirim lain yang menyamar) -- dan
integritas -- bahwa pesan yang sampai ke penerima adalah benar pesan yang
dikirim oleh pengirim tanpa mengalami perubahan sedikitpun.

\subsection{Hill's Cipher}
Pada tahun 1929, seorang matematikawan bernama Lester Hill menciptakan sebuah 
\textit{cryptosystem} polialfabetik. Ide dasar dari Hill's Cipher ini adalah mengambil 
$m$ kombinasi linear dari $m$ karakter alfabet sebagai sebuah elemen dari suatu 
\emph{plaintext}, menghasilkan $m$ karakter alfabet sebagai sebuah elemen dari 
f\emph{ciphertext}-nya.

\subsection{Kriptanalisis}
Kriptanalisis (dari bahasa Inggris \emph{cryptanalysis -- crypto-analysis}) adalah
bagian dari kriptografi yang mempelajari kelemahan suatu algoritma \emph{cryptosystem}
dengan tujuan mendapatkan pesan asli dari suatu \emph{ciphertext} tanpa mengetahui 
kunci yang digunakan untuk enkripsinya. Oleh karena itu, kriptanalisis sering juga 
serangan kriptografi.\par
Berdasarkan model serangannya, beberapa serangan umum adalah:
\begin{enumerate}[label=\alph*.]
	\item \emph{Ciphertext-only attack} (\textbf{COA})\\ 
	Penyerang hanya memiliki \emph{ciphertext} dan tidak memiliki informasi tambahan apapun.
	\item \emph{Known-plaintext attack} (\textbf{KPA})\\
	Penyerang mengetahui sebagian atau seluruh \emph{plaintext} dari \emph{ciphertext} yang dimiliki.
	\item \emph{Chosen-plaintext attack} (\textbf{CPA})\\
	Penyerang bisa mendapatkan \emph{ciphertext} dari \emph{plaintext} yang dimiliki.
\end{enumerate}
Untuk penelitian ini, penulis akan membahas COA dan KPA saja, karena metode CPA bisa dianggap tidak \textit{feasible}.\\

\subsubsection{Analisis Statistik}
Pada \emph{ciphertext-only attack}, penyerang hanya memiliki \emph{ciphertext} tanpa ada informasi
lainnya. Oleh karena itu, petunjuk untuk membongkar pesan hanya bisa didapat dari \emph{ciphertext}
itu sendiri. Salah satu cara untuk mendapatkan petunjuk adalah dengan melakukan  analisis statistik.
Pada kriptografi citra, ada beberapa properti yang bisa dijadikan aspek analisis statistik:
\begin{enumerate}[label=\alph*.]
	\item Histogram\\
Histogram menyatakan persebaran warna dari sebuah citra. Algoritma enkripsi yang baik harus dapat
menghasilkan histogram yang tersebar merata, sekaligus juga berbeda jauh dari histogram \emph{plainimage}-nya.
	\item Koefisien korelasi\\
Koefisien korelasi menyatakan hubungan keterkaitan antara dua variabel. Misalkan kedua variabel
masing-masing adalah \emph{plainimage} dan \emph{cipherimage}; nilai koefisien korelasi yang
mendekati 1 menunjukkan bahwa kedua variabel saling terkait, yang berarti proses enkripsi dapat
dikatakan gagal. Koefisien korelasi mendekati -1 berarti \emph{plainimage} mirip dengan negatif dari
\emph{cipherimage}. Algoritma enkripsi yang baik harus dapat menghasilkan nilai koefisien korelasi
yang mendekati 0.
	\item Entropi informasi\\
Secara umum, entropi menunjukkan ukuran ketidakpastian dari suatu informasi. Makin tinggi entropi
berarti makin tidak pasti suatu informasi. Dalam hubungannya dengan enkripsi, entropi yang tinggi
berarti \emph{cipherimage} tidak dapat diprediksi, sehingga semakin tinggi nilai entropi semakin
baik sebuah algoritma enkripsi.
\end{enumerate}


\section{Matriks}
Matriks merupakan struktur data yang sangat umum dan natural dalam komputasi,
terdiri atas kumpulan angka yang membentuk persegi panjang. Matriks memiliki
beberapa karakteristik penting, yaitu:
\begin{enumerate}[label=\alph*.]
\item Dimensi\\Sebuah matriks dengan $n$ baris dan $m$ kolom disebut sebagai matriks $n\times m$.
Dua bilangan inilah yang disebut sebagai dimensi sebuah matriks. Jika $n = m$, maka
matriks dikatakan sebagai matriks persegi.
\item Elemen\\Setiap angka dalam sebuah matriks merupakan sebuah elemen. Sebuah
elemen $a_{ij}$ dari matriks $A$ terletak pada baris $i$ dan kolom $j$.
\item Vektor\\Dalam hubungannya dengan matriks, sebuah vektor $v$ adalah sebuah
matriks yang berdimensi $1\times n$ atau $n\times 1$.
\[
\begin{bmatrix}
  a_{11} & a_{12} & a_{13} & \cdots & a_{1m} \\
  a_{21} & a_{22} & a_{23} & \cdots & a_{2m} \\
  a_{31} & a_{32} & a_{33} & \cdots & a_{3m} \\
  \vdots  & \vdots  & \vdots & \ddots & \vdots  \\
  a_{n1} & a_{n2} & a_{n3} & \cdots & a_{nm} 
 \end{bmatrix}
 \]
\end{enumerate}

\subsection{Operasi Matriks}
Layaknya angka, matriks pun memiliki beberapa operasi aritmatika. Beberapa
operasi matriks sangat trivial, sisanya memerlukan pengetahuan khusus. Dalam
penelitian ini, penulis akan menggunakan beberapa operasi matriks dasar, yaitu:

\begin{enumerate}[label=\alph*.]

\item Penjumlahan dan pengurangan matriks\\Penjumlahan dan pengurangan matriks
merupakan salah satu operasi trivial. Operasi penjumlahan dan pengurangan
matriks dilakukan elemen per elemen. Syarat penjumlahan dan pengurangan matriks
adalah kedua matriks harus memiliki dimensi yang sama.
\begin{equation} \label{penjumlahan_matriks}
[a_{ij}] \pm [b_{ij}] = [a_{ij}\pm b_{ij}] \text{, untuk setiap }  i \text{ dan } j
\end{equation}

\item Perkalian skalar\\Perkalian skalar juga merupakan operasi trivial. Jika $k$ adalah sebuah
konstanta, maka $kA$ adalah perkalian skalar matriks $A$ dengan $k$,
didefinisikan sebagai matriks dengan elemen-elemen $a_{ij}$ yang
dikalikan dengan k.
\begin{equation}
k[a_{ij}] = [ka_{ij}] \text{, untuk setiap } i \text{ dan }j
\end{equation} 

\item Perkalian matriks\\Jika A adalah matriks $n \times m$ dan B adalah matriks $m \times r$ maka
perkalian matrix $C = A \times B$ didefinisikan sebagai
\begin{equation}
	c_{ij}=\sum_{k=1}^m{a_{ik}b_{kj}}\text{ , } \left(
	\begin{array}{l}
		i=1, 2, 3, \dotsc, n\\
		j=1, 2, 3, \dotsc, r
	\end{array}\right)
\end{equation} 
Matriks $C$ hasil perkalian akan berdimensi $n\times r$.

\subsection{Matriks identitas}
Untuk sebuah bilangan bulat positif $n$, matriks identitas $n \times{n}$,
dinotasikan sebagai $I_n$ atau $I$ saja, adalah matriks 
\begin{equation}
	\left[\delta_{ij}\right]=
	\begin{cases}
		1& \text{jika $i=j$}\\
		0& \text{jika $i \neq j$}
	\end{cases}
\end{equation}
Penamaan ini didasarkan pada sifatnya yang mirip dengan angka 1
sebagai identitas perkalian pada bilangan real: perkalian sebarang
matriks persegi $A$ dengan $I$ menghasilkan $A$.

\subsection{Invers matriks}
Sebuah matriks persegi A dinyatakan \textit{invertible} (memiliki invers) jika dan
hanya jika ada matriks lain $A^{-1}$ yang memenuhi persamaan berikut:
\begin{equation}
	AA^{-1}=A^{-1}A=I
\end{equation}
\end{enumerate}

\section{Operasi bilangan bulat modular}
Bilangan mod(ulus) $m$, $\mathbb{Z}_m$ , adalah himpunan $\lbrace x|x = 0, 1, 2, 3, \dotsc , m - 2, m -
1 \rbrace $. Operasi yang berlaku di domain bilangan bulat berlaku juga di bilangan bulat
mod m. Secara umum, operasi modulus m didefinisikan secara rekursif sebagai
berikut:
\begin{equation}
c \bmod{m}= \begin{cases}
	(m+c)\bmod{m}	&\text{jika }c<0\\
	c				&\text{jika }0\leq c<m\\
	(c-m)\bmod{m}	&\text{jika }c\geq m
\end{cases}
\end{equation}

\subsection{Penjumlahan dan pengurangan}
Penjumlahan dan pengurangan bilangan bulat modulus $m$ didefinisikan sebagai berikut:
\begin{equation}
\begin{array}{lrcl}
	\text{jika}	& a\pm{b}			&=&c\\
	\text{maka}	& (a\pm b)\bmod m	&=&c\bmod m\\
	\;			&\;					&=&(a\bmod m)\pm (b\bmod m)
\end{array}
\end{equation}

\subsection{Perkalian}\
\begin{equation}
	\begin{array}{lrcl}
		\text{jika}	& a\times b			&=&c\\
		\text{maka}	& (a\times b)\bmod m&=&c\bmod m\\
		\;			&\;					&=&(a\bmod m)\times (b\bmod m)\\
	\end{array}
\end{equation}

\subsection{Pembagian (sebagai invers perkalian)}
$a\div b\bmod m=c$ jika dan hanya jika ada $c\in \mathbb{Z}_m$ yang memenuhi persamaan 
$(b\times c)\bmod m=a$.

\subsection{\textit{Multiplicative inverse}}
Invers dari $b$, $b^{-1}$ adalah sebuah bilangan $c\bmod m$ yang memenuhi 
persamaan $(b\times c)\bmod m=1$.\par
Sebuah bilangan $c\bmod m$ memiliki inverse jika $c$ relatif prima (\textit{coprime})
terhadap $m$, yaitu $\operatorname{gcd}(c,m)=1$. Oleh karena itu, penting untuk memilih 
modulo yang benar sehingga sebagian besar $c\in\mathbb{Z}_m$ memiliki invers.

\section{Matriks Modular}
Matriks modular adalah matriks dengan elemen-elemen bilangan modulus.
Matriks modular memiliki aturan-aturan yang hampir sama dengan matriks real
tetapi pada domain yang berbeda, yaitu bilangan modulus. Operasi-operasinya
pun didasarkan pada operasi modulus.
\newpage