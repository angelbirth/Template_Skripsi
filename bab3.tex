%-----------------------------------------------------------------------------%
\chapter{METODOLOGI PENELITIAN}
%-----------------------------------------------------------------------------%

%
\vspace{12pt}
\section{Gambaran Umum}
Penulis akan meneliti implementasi Hill Cipher untuk penyandian citra. Di dalam 
sistem ini, data masukan berupa citra dan matriks kunci yang diberikan pengguna.
Data keluaran berupa citra terenkripsi (\emph{cipherimage}). Pada \emph{cipherimage}
tersebut akan dikenai analisis statistik untuk mengevaluasi seberapa kuat algoritma
Hill Cipher untuk enkripsi citra.
\section{Deskripsi Data}
Data yang penulis gunakan adalah \textit{file} citra yang didapatkan dari kamera
digital. Akan tetapi secara umum batasannya adalah citra keabuan atau berwarna 
tiga kanal (RGB); batasan teknisnya tergantung pada alat (\textit{software}) 
apa yang digunakan.
Dalam penelitian ini penulis menggunakan 5 citra berwarna dan 1 citra keabuan.
Dari kelima citra berwarna, 4 di antaranya diambil dari kamera digital sedangkan 1 citra sisanya
didapat dari internet.

\section{Deskripsi Alat}
Untuk penelitian ini, penulis menggunakan komputer pribadi dengan spesifikasi sebagai berikut:
\begin{itemize}
\item Prosesor AMD A4-3330MX
\item RAM 4GB
\item Kapasitas penyimpanan 500GiB
\end{itemize}
Sedangkan perangkat lunak yang penulis gunakan adalah:
\begin{itemize}
\item Sistem operasi Linux Ubuntu 14.04.5
\item GNU Octave 4.0.2
\item {\LaTeX} dengan paket texlive untuk penulisan dokumen
\end{itemize}

\section{Desain Penelitian}
Dalam penelitian ini, penulis membagi penelitian menjadi beberapa modul utama:
\begin{itemize}
\item Modul \emph{key generation}\\
Modul ini bertugas untuk membuat matriks kunci dari masukan pengguna. Masukan pengguna berupa deretan
karakter dengan panjang maksimal 9 karakter. Matriks kunci berdimensi $3\times{3}$ dan diurutkan
berdasarkan \emph{column-major order}.
\item Modul \emph{inverse key generation}\\
Modul ini bertugas untuk membuat invers dari matriks kunci untuk proses dekripsi citra.
\item Modul enkripsi-dekripsi\\
Modul ini bertugas untuk mengubah \emph{plainimage} menjadi \emph{cipherimage} dan sebaliknya.
\end{itemize}

\section{Desain Pengujian}
Penulis mengambil beberapa aspek dari proses enkripsi-dekripsi sistem sebagai parameter pengujian.
Aspek-aspek tersebut dapat digunakan sebagai tolok ukur kekuatan algoritma enkripsi Hill Cipher. Tahap pengujian juga terbagi menjadi beberapa modul. Setiap aspek ditangani oleh satu modul.
\begin{itemize}
\item Waktu tempuh proses (\emph{running time})
\item Jumlah piksel beda
\item Histogram
\item Koefisien korelasi
\item Entropi informasi
\end{itemize}