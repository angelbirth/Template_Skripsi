%-----------------------------------------------------------------------------%
\chapter{ANALISA DAN PERANCANGAN SISTEM}
%-----------------------------------------------------------------------------%

%
\vspace{12pt}

\section{Gambaran Umum}
Penulis akan meneliti bagaimana cara mengimplementasikan algoritma
enkripsi citra Hill Cipher termodifikasi, sesuai yang diusulkan Panighray dkk.
Akan tetapi, penulis tidak menggunakan \textit{involutory key matrix} seperti yang
digunakan Panighray dkk. dan Acharya dkk.\par
Di dalam sistem ini, masukan akan berupa citra dan kata kunci yang
disuplai oleh pengguna. Masukan ini akan diproses dengan tahapan-tahapan sebagai berikut:
	\begin{enumerate}[label=(\alph*)]
	\item \label{enkripsi1} Pengguna memberikan masukan citra dan matriks kunci $k\bmod 256$.
	\item \label{enkripsi2} Sistem memeriksa apakah citra merupakan citra berwarna atau citra keabuan.
		\begin{enumerate}[label=\arabic*.]
		\item Jika citra berwarna, matriks citra ditransformasikan menjadi matriks 
			berdimensi $3\times n$, dengan $n$ bernilai hasil kali panjang dan lebar citra.
		\item Jika citra keabuan, matriks citra ditransformasikan menjadi matriks 
			berdimensi $3\times\lceil n\rceil$, dengan $n$ bernilai hasil kali 
			panjang dan lebar citra dibagi 3.
		\end{enumerate}
	\item \label{enkripsi3} Matriks citra hasil transformasi dikenai metode Hill Cipher dengan 
		matriks kunci $k$ dan modulo 256.
	\item \label{enkripsi4} Matriks hasil enkripsi dikembalikan ke dimensi asli citranya -- $3\times w\times h$ 
		untuk citra berwarna dan $w\times h$ untuk citra keabuan.
	\end{enumerate}
Untuk dekripsi, tahap-tahapnya adalah sebagai berikut:
	\begin{enumerate}[label=(\alph*)]
	\item Pengguna memberi masukan citra terenkripsi (\textit{cipher image}) dan matriks kunci $k\bmod 256$.
	\item Sistem menghitung invers matriks kunci (yang diasumsikan nonsingular).
	\item Proses selanjutnya sama seperti tahap \ref{enkripsi1} s.d. \ref{enkripsi4} untuk enkripsi,
		dengan mengganti matriks $k$ dengan invers matriks kunci.
	\end{enumerate}
\newpage