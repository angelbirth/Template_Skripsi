%-----------------------------------------------------------------------------%
\chapter{HASIL DAN ANALISIS}
%-----------------------------------------------------------------------------%

%
\vspace{12pt}

\section{Hasil Penelitian}
Pada bab ini akan dibahas hasil dari penelitian yang dilakukan beserta analisisnya. 
Percobaan dilakukan kepada lima gambar beresolusi 24 megapiksel ($6016\times 4000$) dan
satu gambar beresolusi $256\times 256$ piksel. Dari lima gambar tersebut, dua di 
antaranya adalah gambar yang sama -- satu gambar berwarna dan satu gambar \textit{grayscale}.
Penulis menggunakan kata kunci yang sama untuk semua citra yang dienkripsi, yaitu \label{kunci}\texttt{rahasia}.\\
Penulis menggunakan prosentase perbandingan jumlah piksel yang berbeda dan 
waktu komputasi sebagai parameter pengujian. Semakin banyak piksel beda, algoritma enkripsi
dapat dikatakan semakin efektif. Semakin kecil waktu komputasi, algoritma dapat dikatakan semakin efisien.
\subsection{Pembuatan Matriks Kunci}
Kata kunci yang digunakan, \texttt{rahasia}, dikenai \textit{padding} spasi  di akhir
untuk melengkapi jumlah karakter menjadi 9 karakter, diubah menjadi representasi numeriknya 
berdasarkan kode ASCII, lalu disusun ulang menjadi matriks persegi. Penataan mengikuti urutan
\textit{column-major order}. 
\[
\begin{bmatrix}
\texttt{r}&\texttt{a}&\texttt{a}\\
\texttt{a}&\texttt{s}&\_\footnotemark[1]\\
\texttt{h}&\texttt{i}&\_
\end{bmatrix}
\]
Hasil akhir kunci adalah seperti berikut:
\begin{equation}\label{eq:matriks_kunci}
\begin{bmatrix}
114&97 &97\\
97 &115&32\\
104&105&32
\end{bmatrix}
\end{equation}
Determinan matriks \ref{eq:matriks_kunci} adalah $-113967\equiv 209\bmod 256$, yang berarti 
matriks tersebut \textit{invertible}. Sedangkan invers matriksnya adalah
\begin{equation}\label{eq:invers_matriks_kunci}
\begin{bmatrix}
    64&    89&   253\\
   224&    88&   177\\
    65&   198&   101
\end{bmatrix}
\end{equation}
Dapat dibuktikan bahwa matriks \ref{eq:matriks_kunci} dan \ref{eq:invers_matriks_kunci} saling invers:
\begin{equation}
\begin{bmatrix}
	114 & 97  & 97\\
	97  & 115 & 32\\
	104 & 105 & 32
\end{bmatrix}
\begin{bmatrix}
    64&    89&   253\\
   224&    88&   177\\
    65&   198&   101
\end{bmatrix}=
\begin{bmatrix}
   35329&   37888&   55808\\
   34048&   25089&   48128\\
   32256&   24832&   48129
\end{bmatrix}
\end{equation}

\begin{equation}
\begin{bmatrix}
    64&    89&   253\\
   224&    88&   177\\
    65&   198&   101
\end{bmatrix}
\begin{bmatrix}
	114 & 97  & 97\\
	97  & 115 & 32\\
	104 & 105 & 32
\end{bmatrix}=
\begin{bmatrix}
   42241&   43008 &  17152\\
   52480&   50433 &  30208\\
   37120&   39680 &  15873
\end{bmatrix}
\end{equation}

\begin{equation}
\begin{bmatrix}
   35329&   37888&   55808\\
   34048&   25089&   48128\\
   32256&   24832&   48129
\end{bmatrix}\equiv
\begin{bmatrix}
   42241&   43008 &  17152\\
   52480&   50433 &  30208\\
   37120&   39680 &  15873
\end{bmatrix}\equiv
\begin{bmatrix}
   1&  0& 0\\
   0&   1&   0\\
   0&   0&   1
\end{bmatrix} \bmod 256
\end{equation}
Berarti matriks \ref{eq:matriks_kunci} dapat digunakan sebagai matriks kunci untuk enkripsi.

\subsection{Enkripsi Citra Berwarna}
\begin{enumerate}[label=(\alph*)]
	\item Data pertama adalah foto mahasiswa TI Sanata Dharma angkatan 2012 saat bertamu ke 
	rumah Bapak Eka Priyatma. Foto ini diambil pada tanggal 
	21 September 2014 pukul 16.43 WIB, dengan kamera NIKON D3200. Resolusinya $6016\times 4000$ 
	piksel, dengan kedalaman warna 24-bit per piksel (8-bit per komponen warna).\\
	Enkripsi gambar \ref{gbr:plain1} membutuhkan waktu 6,99 detik dengan perbandingan jumlah piksel
	beda 99,615\%\\	
	\begin{figure}\centering
	\begin{subfigure}{\textwidth}
		\includegraphics[width=\textwidth]{data/1comp}
		\caption{Foto mahasiswa TI Sanata Dharma angkatan 2012 saat bertamu ke rumah 
		u	Bapak Drs. Johanes Eka Priyatma, M.Sc., Ph.D.;}
		\label{gbr:plain1}
	\end{subfigure}
	\begin{subfigure}{\textwidth}
		\includegraphics[width=\textwidth]{data/1enc}
		\caption{Hasil enkripsi gambar \ref{gbr:plain1}}
		\label{gbr:enc1}
	\end{subfigure}
	\caption{Data 1} \label{gbr:data1}
	\end{figure}
	Dari gambar \ref{gbr:enc1} terlihat bahwa sebagian besar foto telah tersebar warnanya dengan 
	rata, hanya sebagian kecil area yang memiliki warna seragam (merah). Hal ini disebabkan adanya
	daerah pada gambar yang warnanya sama persis, sehingga hasil enkripsi pun sama persis.
	Hal ini dapat diatasi dengan menyebarkan warna sehingga tidak ada area yang cukup besar dengan 
	warna yang sama.\\
	
	\item Data kedua adalah foto pemandangan di pantai Sadranan, Kabupaten Gunungkidul. Foto
	diambil tanggal 27 Mei 2014 pukul 6.02 WIB dengan kamera NIKON D3200. Resolusi $6016\times 4000$
	piksel, 24-bit warna per piksel.
	\begin{figure}\centering
		\begin{subfigure}{\textwidth}
		\includegraphics[width=\textwidth]{data/2comp}
		\caption{}
		\label{gbr:plain2}
	\end{subfigure}
	\begin{subfigure}{\textwidth}
		\includegraphics[width=\textwidth]{data/2enc}
		\caption{Hasil enkripsi gambar \ref{gbr:plain2}}
		\label{gbr:enc2}
	\end{subfigure}
	\caption{Data 2}
	\label{gbr:data2}
	\end{figure}
	
	\item Citra ketiga
	\begin{figure}\centering
		\begin{subfigure}{\textwidth}
		\includegraphics[width=\textwidth]{data/3comp}
		\caption{}
		\label{gbr:plain3}
	\end{subfigure}
	\begin{subfigure}{\textwidth}
		\includegraphics[width=\textwidth]{data/3enc}
		\caption{Hasil enkripsi gambar \ref{gbr:plain3}}
		\label{gbr:enc3}
	\end{subfigure}
	\caption{Data 3}
	\label{gbr:data3}
	\end{figure}
	
	\item Citra keempat
	\begin{figure}\centering
	\begin{subfigure}{\textwidth}
		\includegraphics[width=\textwidth]{data/4comp}
		\caption{}
		\label{gbr:plain4}
	\end{subfigure}
	\begin{subfigure}{\textwidth}
		\includegraphics[width=\textwidth]{data/4enc}
		\caption{Hasil enkripsi gambar \ref{gbr:plain4}}
		\label{gbr:enc4}
	\end{subfigure}
	\caption{Data 4}
	\label{gbr:data4}
	\end{figure}
	
	\item Citra kelima
	\begin{figure}\centering
		\begin{subfigure}{\textwidth}
			\includegraphics[width=256px]{data/5}
			\caption{}
			\label{gbr:plain5}
		\end{subfigure}
		\begin{subfigure}{\textwidth}
			\includegraphics[width=256px]{data/5enc}
			\caption{Hasil enkripsi gambar \ref{gbr:plain5}}
			\label{gbr:enc5}
		\end{subfigure}
		\caption{Data 5}
		\label{gbr:data5}
	\end{figure}
\end{enumerate}

\begin{table} \centering
\caption{Hasil uji citra}
	\begin{tabular}{|r|c|c|c||} \hline
	\textbf{No Gambar} &{\bf Resolusi}    &\textbf{Waktu\footnotemark[2]}&\textbf{Perbedaan}\\ \hline
	\ref{gbr:data1}    &$6016\times 4000$ &6.99 detik    &99.615\%          \\ \hline
	\ref{gbr:data2}    &$6016\times 4000$ &7.50 detik    &99.616\%          \\ \hline
	\ref{gbr:data3}    &$6016\times 4000$ &7.30 detik    &99.627\%          \\ \hline
	\ref{gbr:data4}    &$6016\times 4000$ &7.56 detik    &99.599\%          \\ \hline
	\ref{gbr:data5}    &$256\times 256$   &0.0213 detik  &99.633\%          \\ \hline
	\end{tabular}
\end{table}

\subsection{Enkripsi Citra Keabuan}
Citra yang dipakai adalah gambar \ref{gbr:plain3} yang diubah menjadi \textit{grayscale}.
 \textit{Property} gambar \ref{gbr:plain6} sama persis dengan gambar \ref{gbr:plain3}, kecuali
 kedalaman warnanya; gambar \ref{gbr:plain6} memiliki kedalaman warna 8-bit per piksel (256 warna)
\begin{figure}\centering
	\begin{subfigure}{\textwidth}
		\includegraphics[width=\textwidth]{data/6comp}
		\caption{}
		\label{gbr:plain6}
	\end{subfigure}
	\begin{subfigure}{\textwidth}
		\includegraphics[width=\textwidth]{data/6enc}
		\caption{Hasil enkripsi gambar \ref{gbr:plain6}}
		\label{gbr:enc6}
	\end{subfigure}
	\caption{Data 6}
	\label{gbr:data6}
\end{figure}

\section{Analisis Pengujian}
\subsection{Analisis Histogram}
\subsection{Koefisien Korelasi}
\subsection{Analisis Entropi}

\footnotetext[1]{Karakter \texttt{\_} dipakai sebagai pengganti spasi supaya lebih terlihat}
\footnotetext[2]{Waktu proses diukur ketika \textit{Octave engine} sudah "panas" -- sudah 
digunakan lebih dari sekali -- sehingga komputasi sudah optimal. Berdasarkan percobaan,
\textit{Octave engine} sudah panas setelah 3 kali memproses data. Waktu yang diambil adalah
waktu proses yang ke-4, yang kebetulan selalu yang terbaik dari antara proses-proses sebelumnya.}


\newpage