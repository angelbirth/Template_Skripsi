%-----------------------------------------------------------------------------%
\chapter{PENUTUP}
%-----------------------------------------------------------------------------%

%
\vspace{12pt}

\section{Kesimpulan}
Secara umum, algoritma Hill Cipher cukup baik untuk digunakan sebagai metode enkripsi citra.
Karena sifat dasarnya sebagai linear kombinator, Hill Cipher mampu menyebarkan informasi
(dalam hal ini, intensitas warna) sehingga cukup resisten terhadap \emph{statistical attack}.
Akan tetapi, karena sifat linearitasnya itu juga, Hill Cipher menjadi lemah terhadap \emph{known-
plaintext attack}. Akan tetapi kelemahan ini bisa dikurangi dengan menambah panjang kunci; semakin 
panjang kuncinya, akan semakin sulit melakukan \emph{known-plaintext attack}. Namun demikian, pertambahan
panjang kunci berimbas pada makin lamanya proses enkripsi (sekaligus dekripsi) dan pembuatan invers 
matriks kunci.

\section{Saran}
Penelitian ini tidak mencakup aspek transmisi dari citra terenkripsi, sehingga pada prakteknya dapat
terjadi adanya \emph{transmission error} yang berakibat citra yang diterima tidak sama dengan citra 
yang ditransmisikan. Selain itu tidak menutup kemungkinan juga bahwa pengguna lupa sebagian atau 
bahkan seluruh kunci yang berakibat pada citra yang tidak bisa terdekripsi dengan baik (atau malah 
tidak dapat terdekripsi sama sekali). Oleh karena itu beberapa hal yang menjadi saran untuk penelitian
selanjutnya:
\begin{itemize}
\item Bagaimana cara transmisi \emph{cipherimage} sehingga tidak ada \emph{information loss}.
\item Bagaimana cara memperbaiki \emph{cipherimage} yang korup sehingga masih dapat didekripsi sebagian.
\item Bagaimana cara memperbaiki kunci yang dilupakan sebagian sehingga masih ada informasi yang bisa didapat dari proses dekripsi.
\item Bagaimana cara penyimpanan dan distribusi kunci sehingga hanya orang yang berhak yang dapat memilikinya.
\end{itemize}
Selain itu, untuk khalayak umum penulis memberi saran:
\begin{itemize}
\item Jangan mengirim informasi sensitif melalui jalur tidak aman. Sekuat apapun algoritma enkripsi tidak
akan mampu mengkompensasi kerugian yang timbul atas kecerobohan manusia.
\item Ketika menyangkut informasi sensitif, jangan percaya siapapun.
\item Pertimbangkan juga untuk menggunakan steganografi yang dikombinasikan dengan kriptografi untuk
mengurangi ketertarikan pihak ketiga.
\end{itemize}

\newpage